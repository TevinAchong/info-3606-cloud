\documentclass[a4paper, 12pt]{article}

\usepackage[left=2cm, right=2cm, top=2cm]{geometry}
\usepackage{color}
\usepackage{graphicx}
\usepackage{float}

\begin{document}
\pagenumbering{roman}
\title{
		\textbf{Student Name:} Tevin Achong\\
		\textbf{Student ID:} 816000026\\
		\textbf{Course Code:} INFO3606\\
		\textbf{Course Title:} Cloud Computing\\
		\textbf{Assignment:} 3
		\date{April 15, 2020}
}
\maketitle

\newpage
\pagenumbering{arabic}

\begin{center}
	\textbf{Question 1}
\end{center}



\newpage
\begin{center}
	\textbf{Question 2}
\end{center}

\newpage
\begin{center}
	\textbf{Question 3}
\end{center}

Cloud service providers typically perform two major kinds of monitoring:
\begin{itemize}
\item
Physical Monitoring
\item
Virtual Monitoring
\end{itemize}

\textbf{Physical Monitoring}
\begin{itemize}
\item
Physical monitoring involves monitoring the physical infrastructure of the cloud. It deals with the internal status of cloud resources. 
\item
From the cloud point of view it refers to the server level. 
\item
Physical system monitoring is the basis for cloud system management as it is related to the physical computing infrastructure on which the cloud itself is built.
\item
It is here that conventional monitoring techniques such as computation-based tests and network-based tests are carried out.
\item
Computation-based tests are based on classical statistical measures such as mean, median, mode and temporal characteristics.
\item
Metrics measured by computation-based tests include: CPU utilization; CPU latency; CPU speed; CPU-related errors; memory utilization; memory page exchanges per second; memory latency; memory-related errors; and operating system metrics like system load;
\item
Metrics measured by network-based tests include: network latency; network utilization; network traffic; load patterns; network capacity bandwith; throughput; response time; round trip delay; jitter; packet/data loss; available bandwith; capacity, and; traffic volume
\end{itemize}

\textbf{Virtual Monitoring}
\begin{itemize}
\item
Virtual system monitoring provides information about the virtual system characteristics of cloud services:
\begin{itemize}
\item
In IaaS cloud the system manager can monitor the status of every VM instance in the cloud and its internal resources. 
\item
In PaaS cloud the system managar can monitor the use of platform resources such as hosting space used and simultaneous network connections.
\item
In SaaS cloud the system manager can mointor application usage patterns, resources shared among applications, etc.
\item
Virtual monitoring is the main means of controlling quality of service metrics and ensuring service delivery according to the values mentioned in SLAs.
\end{itemize}
\item
Service providers use the metrics obtained from virtual system monitoring to calculate service costs and billing.
\item
Successful cloud business models almost always stem from virtual monitoring.
\item
Cloud users perform high-level monitoring determine how many resources they have consumed so that they can check their bills.
\item
In addition, such high-level monitoring makes it easy for consumers to compare the pricing of different cloud service providers.
\item

\end{itemize}


\textbf{Purpose of Monitoring the Cloud:}
\begin{itemize}
\item
Since there are physical and virtual resources that are distributed at different geographical locations in the cloud, without proper monitoring of cloud resources and services enterprises may fail to achieve the performance they set for themselves and benefits provided by the cloud. 
\item
They run the risk of failing to achieve the return on investment targeted.
\item
Cloud monitoring is the process of monitoring, evaluating, and managing cloud-based services, applications, and infrastructure.
\item
Aditionally, cloud monitoring is fundamental to:
\begin{itemize}
\item
capacity planning
\item
calculating the usage of resources and providing billing to consumers
\item
identifying and addressing potential issues and providing troubleshooting
\item
delivering services according to the service level agreement
\item
providing detailed reports, in-depth graphs, and different metrics for performance management
\item
optimizing cloud solutions and services
\item
determining the status of resources hosted in the cloud
\item
adopting control activities and performing various core activities of the cloud such as resource allocation
\item
migrating without data loss
\item
managing security
\item
achieving those characteristics that are unique to the cloud: scalability, elasticity, and resource provisioning
\end{itemize}
\end{itemize}


\newpage
\begin{center}
	\textbf{Question 4}
\end{center}

Six (6) core functions of Cloud Management Platforms (CMPs):
\begin{enumerate}
\item
Service Request Management
\begin{itemize}
\item
This is a self-service interface provided by CMPs through which various cloud services are easily consumed by consumers. 
\item
Cloud services providers offer service catalogs with SLAs and cost details.
\item
Based on the published information the CMP chooses the appropriate provider and services.
\item
Service requests can be routed through this interface to the CMP solution to automate most activities.
\item
Some users expect a service interface that serves as a pass-through to native capabilities within a public cloud service.
\item
The service portal or marketplace is continuously updated with fresh features, functionalities, and facilities to gain an edge or retain the edge gained.
\item
There are service and support management systems and other automation tools that readily fulfill varying requests from users.
\item
There are operational team members employed by cloud service providers or third-party teams that team up together to fulfill service requests quickly.
\end{itemize}

\item
Provisioning, Orchestration, and Automation
\begin{itemize}
\item
Provisioning, orchestration, and automation are the core capabilities of any CMP product.
\item
There is an arsenal of tools intrinsically enabling these vital features. There are plenty of cloud orchestration, provisioning, and configuring tools that are made available these days.
\item
There are industry strength standards for service and cloud infrastructure orchestration. Similarly, there are automation tools for job/task scheduling, load balancing, auto-scaling, resource allocation, etc.
\item
There are resource configuration management systems. Software deployment and delivery tools are also hitting the market.
\item
In a nutshell, cloud operations are being meticulously automated in an end-to-end fashion. 
\end{itemize}

\item
Monitoring and Metering
\begin{itemize}
\item
Monitoring, measurement, management, and metering are the basic requirements of any IT hardware and software packages.
\item
Service usage and resource consumption need to be accurately measured and metered. There are a bunch of tools for accomplishing these.
\end{itemize}

\newpage
\item
Multi-Cloud Brokering
\begin{itemize}
\item
Brokerage solutions and services are very important at a time dominated by connected and federated clouds.
\item
Interconnectivity, intermediation, and other enrichment and enablement capabilities are being performed through cloud service brokers.
\item
There are connectors, adapters, drivers, and other solutions that establish a seamless linkage between public and private clouds.
\item
There are bridge solutions to establish direct connectivity between public clouds.
\item
Thus as a result of multiple clouds and services with different SLAs the role and responsibility of cloud brokers are bound to increase in the days ahead. Advanced CMPs are being fitted with brokerage tools and engines.
\end{itemize}

\item
Security and Identity
\begin{itemize}
\item
Concern over security is widespread among cloud users who rightly insist on security requirements in the cloud environment. 
\item
As customer-facing applications and data (corporate, customer, and confidential) are being held in the cloud environment, especially in public cloud, security is paramount. User identification, authentication, authorization, and other accountability and auditability are being pronounced as the most critical and crucial for continued spread of the cloud.
\item
The security and privacy of data while in transit, persistence, and usage are paramount for the intended success of the cloud idea.
\item
Key-based encryption and decryption, key management, etc. are getting a lot of attention these days. Single sign-on (SSO) is indispensable to multi-cloud applications.
\item
United threat and vulnerability management solutions are becoming highly popular in the cloud environment.
\end{itemize}

\item
Service-Level Management
\begin{itemize}
\item
Ensuring service-level and operation-level contracts agreed between cloud consumer and server are complied with is an important facet of the cloud arena. 
\item
Non-functional requirements (NFRs)/quality of service (QoS) attributes stand out as the key differentiators among all participating service providers.
\item
Scalability, availability, fault tolerance, security, and dependability are often repeated needs.
\item
Service resilience, application readability, and infrastructure versatility are given utmost importance for boosting user confidence in the cloud "mystery".
\item
There is a stunning array of toolsets that can be used to facilitate these complex capabilities. 
\end{itemize}
\end{enumerate}


\newpage
\begin{center}
	\textbf{Question 5}
\end{center}

Five (5) reasons why Cloud Service Brokerages (CSBs) are needed:
\begin{enumerate}
\item
Cloud service brokerages help organizations select the best cloud services for that organization's needs.
\item
Cloud service brokerages support line-of-business requirements, and meet IT demands across disparate cloud without jeopradizing performance or security.
\item
Cloud service brokerages may assist in the deployment and integration of applications across multiple clouds.
\item
Cloud service brokerages may provide options and possible cost savings by providing a catalog that compares competing services.
\item
Cloud service brokerages are master orchestrators who can manage the complexity of multiple cloud ecosystems and transform businesses into digital enterprises.
\end{enumerate}

Cloud service brokerage serves as an intermediary between cloud providers and cloud consumers. It assists companies in choosing services and offerings that best suit their needs. Cloud technology is increasingly playing a major part in bringing about digital business.
The rapid adoption of cloud services from multiple cloud service providers (CSPs) and communication service providers creates a unique set of challenges for IT, specifically because enterprise IT teams must now orchestrate onboarding, managing, and delivering IT and business services from multiple portals and vendors. Such multiplicity makes it tough to ensure consistent performance, security, and control within the multi-cloud ecosystem and is the reason cloud brokerage platform solutions are becoming popular.

\newpage
\begin{center}
	\textbf{Question 6}
\end{center}

Cloud Orchestration
\begin{itemize}
\item
Orchestration is concerned with automating multiple tasks together. Processes typically comprise multiple tasks and systems.
\item
The tasks inscribed in a process need to be executed in sequence to be fruitful.
\item
That is, a process starts with an appropriate workflow representation and end with workflow execution. 
\item
Thus a process from workflow representation to workflow execution is simply termed orchestration.
\item
Orchestration deals with the end-to-end process, including management of all related services, taking care of high availability (HA), post-deployment, failure recovery, scaling and more. 
\item
Automating tasks or orchestration of workflows within a single enterprise may be easier as all the services, such as APIs, interfaces, standards, regulations, and policies, are confined within the enterprise. 
\item
However, enterprises now find themselves compelled to look at cloud offerings to meet many critical needs such as reduced capital and operational cost, uncertain loads, dynamic and unlimited scalability needs, and high availability. They depend on public clouds because of their inherent capabilities. At the same time enterprises have to depend on on-premises setups to protect legacy and sensitive data. 
\item
In spite of the challenges associated with a multi-cloud environment, enterprises are compelled to opt for multi-cloud and hybrid IT because of their benefits. 
\item
Cloud orchestration resolves some of the challenges associated with operations in multi-cloud environment. 
\end{itemize}

Four (4) functions of the Orchestration Engine:
\begin{enumerate}
\item
helps automate delivery of infrastructure, application, and custom IT services
\item
supports direct integration of service management capabilities
\item
deploys application workloads across on-premises and off-premises environments
\item
provides policy-based governance and logical application modeling to help ensure that multi-vendor and multi-cloud services are delivered at the right size and service level for each task performed.
\end{enumerate}
\end{document}